\documentclass{article}
\usepackage[spanish]{babel}
\usepackage[utf8]{inputenc}
\usepackage{hyperref}
\providecommand{\keywords}[1]{\textbf{\textit{Palabras Clave---}} #1}

\begin{document}
	\title{MÓDULO 6 - Proyecto Final}
	\author{David López Pretel}
	\maketitle
	\newpage
	\tableofcontents
	\newpage
	
	\begin{abstract}
		El objetivo de la detección de anomalías es identificar las observaciones que difieren significativamente de la mayoría de los datos.
		La detección de anomalías se aplica con frecuencia a las series temporales, que son datos con un componente temporal. Existen varias formas de abordar un problema de detección de anomalías para series temporales, sin embargo, hay muy pocas opciones para los problemas de detección de anomalías en series temporales multivariantes. Las técnicas con un mayor rendimiento en la tarea de detección de anomalías en series temporales son: Redes Neuronales Temporales (TNN) y Redes Neuronales Recurrentes (RNN).\\				\url{https://github.com/DLPretel/proyecto_final}
	\end{abstract}

	\keywords{Detección de Anomalías, Series Temporales, Redes Neuronales}
	
	\section{Introduccion}
	
	
	
	\section{Estado del arte}
	
	
	
	\bibliography{M6_Lopez_Pretel_David}
	\bibliographystyle{plain}
	
\end{document}